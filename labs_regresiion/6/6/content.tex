\section{Лістинг програми}

\lstinputlisting[language=Python]{../6.py}

\newpage

\section{Результат роботи програми}

Функція якою задаються точки плану:
\[f(x_1,x_2,x_3) = 8.8+8.0\cdot x_1+5.4\cdot x_2+8.0\cdot x_3+0.2\cdot x^2_1+0.2\cdot x^2_2+2.9\cdot x^2_3+3.4\cdot x_1\cdot x_2+0.9\cdot x_1\cdot x_3+3.5\cdot x_2\cdot x_3+\]

$+0.3\cdot x_1\cdot x_2 \cdot x_3$

Нормована матриця планування:

\begin{center}
    \tiny
    \begin{tabular}{|c|c|c|c|c|c|c|c|c|c|c|c|c|c|}
        \hline
         & $X_{1}$ & $X_{2}$ & $X_{3}$ & $X_{1}\cdot X_{2}$ & $X_{1}\cdot X_{3}$ & $X_{2}\cdot X_{3}$ &$X_{1}\cdot X_{2}\cdot X_{3}$ & $X^2_{1}$&$X^2_{2}$ &$X^2_{3}$ &$Y_{1}$ & $Y_{2}$ & $Y_{3}$ \\ 
         \hline
         1& -0.58 & -0.58 & -0.58 & 0.3364 & 0.3364 & 0.3364 & -0.195 & 0.3364 & 0.3364 & 0.3364 & 4.0635 & 8.500 & -3.197\\
         2& -0.58 & -0.58 & 0.58 & 0.3364 & -0.3364 & -0.3364 & 0.195 & 0.3364 & 0.3364 & 0.3364 & -3.936 & 3.500 & 4.802\\
         3& -0.58 & 0.58 & -0.58 & -0.3364 & 0.3364 & -0.3364 & 0.195 & 0.3364 & 0.3364 & 0.3364 & 4.063 & 1.500 & 5.802\\
         4& -0.58 & 0.58 & 0.58 & -0.3364 & -0.3364 & 0.3364 & -0.195 & 0.3364 & 0.3364 & 0.3364 & 3.063 & 2.500 & -0.197\\
         5& 0.58 & -0.58 & -0.58 & -0.3364 & -0.3364 & 0.3364 & 0.195 & 0.3364 & 0.3364 & 0.3364 & -4.936 & 7.500 & 4.802\\
         6& 0.58 & -0.58 & 0.58 & -0.3364 & 0.3364 & -0.3364 & -0.195 & 0.3364 & 0.3364 & 0.3364 & 0.063 & 2.500 & 2.802\\
         7& 0.58 & 0.58 & -0.58 & 0.3364 & -0.3364 & -0.3364 & -0.1958 & 0.3364 & 0.3364 & 0.3364 & 2.063 & 9.500 & -2.197\\
         8& 0.58 & 0.58 & 0.58 & 0.3364 & 0.3364 & 0.3364 & 0.195 & 0.3364 & 0.3364 & 0.3364 & 3.064 & 5.500 & -3.197\\
         8& -1.0 & 0.0 & 0.0 & -0.0 & -0.0 & 0.0 & -0.0 & 1.0 & 0.0 & 0.0 & -3.936 & 6.500 & -3.197\\
         10& 1.0 & 0.0 & 0.0 & 0.0 & 0.0 & 0.0 & 0.0 & 1.0 & 0.0 & 0.0 & 2.063 & 7.500 & 4.802\\
         11& 0.0 & -1.0 & 0.0 & -0.0 & 0.0 & -0.0 & -0.0 & 0.0 & 1.0 & 0.0 & 4.063 & 10.500 & 1.802\\
         12& 0.0 & 1.0 & 0.0 & 0.0 & 0.0 & 0.0 & 0.0 & 0.0 & 1.0 & 0.0 & -0.936 & 10.500 & -2.197\\
         13& 0.0 & 0.0 & -1.0 & 0.0 & -0.0 & -0.0 & -0.0 & 0.0 & 0.0 & 1.0 & 1.06 & 6.500 & 5.802\\
         14& 0.0 & 0.0 & 1.0 & 0.0 & 0.0 & 0.0 & 0.0 & 0.0 & 0.0 & 1.0 & 1.06 & 10.500 & -0.197\\
        \hline
    \end{tabular}
\end{center}
\normalsize
Далі було знайдено нормовані коефіцієнти рівняння:
\begin{center}
    $y = 0.884+0.006\cdot x_{n1}+0.11\cdot x_{n2}-0.028\cdot x_{n3}-0.047\cdot x_{n1}\cdot x_{n2}-4.533e-5\cdot x_{n1}\cdot x_{n3}-$
    $-3.05e-4\cdot x_{n2}\cdot x_{n3}+9.79e-5\cdot x_{n1}\cdot x_{n1}\cdot x_{n3}+0.003\cdot x^2_{n1}+0.003\cdot x^2_{n2}-0.002\cdot x^2_{n3}$        
\end{center}

Критерій Кохрена показав наступне значення:
\[G_p = 0.153\]

Вищезгадане значення менше за $G_T=0.201$, а отже дисперсія однорідна.

Критерій Стьюдента показав наступні значення:
\small
\[t_0=10.3,\; t_1=2.7,\; t_2=2.1,\; t_3=2.5,\; t_4=1.0,\; t_5=0.02,\; t_6=0.14,\; t_7=0.08,\; t_8=2.8,\; t_9=3.7,\; t_{10}=3.8\]
\normalsize
Значення $t_{4,5,6,7}<2.048$, тому коефіцієнти рівняння регресії приймаємо
незначними при рівні значимості 0.05

Таким чином рівняння регресії має вигляд:
\[y = 0.884+0.006\cdot x_{n1}+0.11\cdot x_{n2}-0.028\cdot x_{n3}+0.003\cdot x^2_{n1}+0.003\cdot x^2_{n2}-0.002\cdot x^2_{n3}\]

Також було знайдено натуралізовані коефіцєнти рівнняння регресії:
\[y = 10.27+1.13\cdot x_{1}+0.52\cdot x_{2}+1.01\cdot x_{3}+0.79\cdot x^2_{1}+1.02\cdot x^2_{2}+1.05\cdot x^2_{3}\]

Останнім кроком була перевірка адекватносі моделі за допомогою критерію Фішера:
\[F_p = 2.37\]

Оскільки $F_p<3.0$, отже рівняння регресії адекватно оригіналу при рівні 
значимості 0.05

\section{Висновки}
В ході даної лабораторної роботи було проведено трьохфакторний експеримент з 
використанням рівняння регресії з урахуванням квадратичних членів. Спочатку 
було згенеровано матрицю плану за заданоъ функції, після чого було перевірено 
однорідність десперсії за критерієм Кохрена. Після цього було знайдено натуралізовані
коефіцієнти, та визначено значимість коефіцієнтів за допомогою критерію Стьюдента, 
який показав, що один з коефіцієнтів є незначним. Адекватность рівняння оригіналу 
було перевірено за допомогою критерію Фішера, який показав, що рівнняння є 
адекватним оригіналу.
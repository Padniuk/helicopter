\section{Лістинг програми}

\lstinputlisting[language=Python]{../2.py}

\section{Результат роботи програми}

Нормована матриця планування:
\begin{center}
    \begin{tabular}{|c|c|c|c|c|c|c|c|}
        \hline
         & $X_{N1}$ & $X_{N2}$ & $Y_{1}$ & $Y_{2}$ & $Y_{3}$ & $Y_{4}$ & $Y_{5}$ \\ 
        \hline
        1 & 1.0 & 1.0 & -854 & -922 & -922 & -873 & -851 \\
        2 & -1.0 & -1.0 & -861 & -877 & -891 & -861 & -932 \\
        3 & -0.2 4 & 0.08 & -866 & -922 & -878 & -892 & -941 \\
        \hline
    \end{tabular}
\end{center}

Критерій Романовського показав наступні значення для $R_{uv}$:
\[R_{uv1}=0.16,\; R_{uv2}=0.18,\; R_{uv3}=0.20\]

Кожне з вищезгаданих значень менше за $R_{cr}=2$, а отже дисперсія однорідна.

Далі було знайдено нормовані коефіцієнти рівняння:
\[y = -884.39+48.12\cdot x_{n1}-48.12\cdot x_{n2}\]

Після чого тривіальною підстановкою нормованих значень в дане рівнняння, було 
перевірено рівність отриманих значень та середніх значень для i-ї точки.

Останнім кроком було знаходження натуралізованих коефіцієнтів:
\[y = -932.52+2.41\cdot x_{1}-1.20\cdot x_{2}\]

Після чого перевірка знову ж показала правильність знаходених коефіцієнтів.
\section{Висновки}
В ході даної лабораторної роботи було проведено двофакторний експеримент з 
використанням лінійного рівняння регресії. Було перевірено однорідність десперсії
закритерієм Романовського після чого було знайдено нормовані та натуральні коефіцієнти
рівняння та перевірено їх правильність за допомогою середніх значень
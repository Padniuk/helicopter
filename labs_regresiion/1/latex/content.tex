\section{Лістинг програми}

\lstinputlisting[language=Python]{../1.py}

\section{Результат роботи програми}

Нормований план експерименту:

\[ X_{N1} = \left(-1.0,\; 0.83,\; -1.0,\; -1.0,\; 0.17,\; 0.83,\; 1.0,\; 0.17\right)\]

\[ X_{N2} = \left(-0.75,\; -1.0,\; 1.0,\; -0.75,\; -0.12,\; 0.38,\; -0.12,\; 0.12\right)\]

\[ X_{N3} = \left(1.0,\; 0.18,\; 0.88,\; -0.53,\; 0.88,\; -1.0,\; 0.88,\; 0.88\right)\]

Функція відгуку від нульових рівнів факторів:

\[Y_{et} = 99.0\]

Критерій для функції відгуку:
\[\rightarrow \bar{Y}\]

Точка, що задовільняє даному критерію:
\[\left(-1.0, -0.75, 1.0\right)\]

Значення плану в даній точці:
\[Y = 103.0\]

\section{Висновки}
В ході даної лабораторної роботи було опановано основні принципи теорії планування експерименту. 
Отримані знання було закріплено практичними використанням: нормування факторів, знаходження нульових рівнів, 
а також знаходження еталонного значення плану та точки плану що задовільняє заданий критерій
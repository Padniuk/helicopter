\section{Лістинг програми}

\lstinputlisting[language=Python]{../3.py}

\newpage

\section{Результат роботи програми}

Нормована матриця планування:
\begin{center}
    \begin{tabular}{|c|c|c|c|c|c|c|}
        \hline
         & $X_{N1}$ & $X_{N2}$ & $X_{N3}$ & $Y_{1}$ & $Y_{2}$ & $Y_{3}$ \\ 
        \hline
        1& -0.62 & 0.26 & 0.38 & 193 & 237 & 228\\
        2& -0.05 & -0.47 & 1.0 & 195 & 241 & 203\\
        3& 0.43 & 1.0 & -0.46 & 195 & 232 & 228\\
        4& -1.0 & -1.0 & 1.0 & 213 & 209 & 192\\
        5& 1.0 & -0.5 & -1.0 & 200 & 206 & 227\\
        \hline
    \end{tabular}
\end{center}

Далі було знайдено нормовані коефіцієнти рівняння:
\[y = 207.45+0.12\cdot x_{n1}+0.21\cdot x_{n2}+0.12\cdot x_{n3}\]

Критерій Кохрена показав наступне значення:
\[G_p = 0.736\]

Вищезгадане значення менше за $G_T=0.321$, а отже дисперсія однорідна.

Критерій Стьюдента показав наступні значення:
\[t_0=52.12,\; t_1=2.26,\; t_2=6.59,\; t_3=9.27\]

Значення $t_1<2.306$, тому коефіцієнт рівняння регресії приймаємо
незначними при рівні значимості 0.05

Таким чином рівняння регресії має вигляд:
\[y = 207.45+0.21\cdot x_{n2}+0.34\cdot x_{n3}\]

Було перевірка адекватносі моделі за допомогою критерію Фішера:
\[F_p = 0.65\]

Оскільки $F_p<4.5$, отже рівняння регресії адекватно оригіналу при рівні 
значимості 0.05

Також було знайдено натуралізовані коефіцієнти рівнянн регресії:
\[y = 213.26+2501.0\cdot x_{2}-784.93\cdot x_{3}\]

\section{Висновки}
В ході даної лабораторної роботи було проведено трьохфакторний експеримент з 
використанням лінійного рівняння регресії. Було перевірено однорідність десперсії
за критерієм Кохрена. Після цього було знайдено натуралізовані та нормалізовані коефіцієнти, та визначено
значимість коефіцієнтів за допомогою критерію Стьюдента, який показав, що один з 
коефіцієнтів є незначним. Адекватность рівняння оригіналу було перевірено за допомогою
критерію Фішера, який показав, що рівнняння є адекватним оригіналу.
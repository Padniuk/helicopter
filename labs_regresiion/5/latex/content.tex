\section{Лістинг програми}

\lstinputlisting[language=Python]{../5.py}

\newpage

\section{Результат роботи програми}

Нормована матриця планування:

\begin{center}
    \tiny
    \begin{tabular}{|c|c|c|c|c|c|c|c|c|c|c|c|c|c|}
        \hline
         & $X_{N1}$ & $X_{N2}$ & $X_{N3}$ & $X_{N1}\cdot X_{N2}$ & $X_{N1}\cdot X_{N3}$ & $X_{N2}\cdot X_{N3}$ &$X_{N1}\cdot X_{N2}\cdot X_{N3}$ & $X^2_{N1}$&$X^2_{N2}$ &$X^2_{N3}$ &$Y_{1}$ & $Y_{2}$ & $Y_{3}$ \\ 
        \hline
        1& -0.58 & -0.58 & -0.58 & 0.34 & 0.34 & 0.34 & -0.19 & 0.34 & 0.34 & 0.34 & 198 & 198 & 195\\
        2& -0.58 & -0.58 & 0.58 & 0.34 & -0.34 & -0.34 & 0.19 & 0.34 & 0.34 & 0.34 & 204 & 202 & 199\\
        3& -0.58 & 0.58 & -0.58 & -0.34 & 0.34 & -0.34 & 0.19 & 0.34 & 0.34 & 0.34 & 203 & 203 & 195\\
        4& -0.58 & 0.58 & 0.58 & -0.34 & -0.34 & 0.34 & -0.19 & 0.34 & 0.34 & 0.34 & 197 & 203 & 199\\
        5& 0.58 & -0.58 & -0.58 & -0.34 & -0.34 & 0.34 & 0.19 & 0.34 & 0.34 & 0.34 & 200 & 202 & 202\\
        6& 0.58 & -0.58 & 0.58 & -0.34 & 0.34 & -0.34 & -0.19 & 0.34 & 0.34 & 0.34 & 204 & 201 & 204\\
        7& 0.58 & 0.58 & -0.58 & 0.34 & -0.34 & -0.34 & -0.19 & 0.34 & 0.34 & 0.34 & 202 & 197 & 201\\
        8& 0.58 & 0.58 & 0.58 & 0.34 & 0.34 & 0.34 & 0.19 & 0.34 & 0.34 & 0.34 & 204 & 198 & 201\\
        9& -1.0 & 0.0 & 0.0 & -0.0 & -0.0 & 0.0 & -0.0 & 1.0 & 0.0 & 0.0 & 196 & 197 & 201\\
        10& 1.0 & 0.0 & 0.0 & 0.0 & 0.0 & 0.0 & 0.0 & 1.0 & 0.0 & 0.0 & 204 & 203 & 204\\
        11& 0.0 & -1.0 & 0.0 & -0.0 & 0.0 & -0.0 & -0.0 & 0.0 & 1.0 & 0.0 & 195 & 201 & 202\\
        12& 0.0 & 1.0 & 0.0 & 0.0 & 0.0 & 0.0 & 0.0 & 0.0 & 1.0 & 0.0 & 202 & 195 & 197\\
        13& 0.0 & 0.0 & -1.0 & 0.0 & -0.0 & -0.0 & -0.0 & 0.0 & 0.0 & 1.0 & 199 & 196 & 200\\
        14& 0.0 & 0.0 & 1.0 & 0.0 & 0.0 & 0.0 & 0.0 & 0.0 & 0.0 & 1.0 & 203 & 197 & 196\\
        15& 0.0 & 0.0 & 0.0 & 0.0 & 0.0 & 0.0 & 0.0 & 0.0 & 0.0 & 0.0 & 198 & 204 & 195\\
        \hline
    \end{tabular}
\end{center}
\normalsize
Далі було знайдено нормовані коефіцієнти рівняння:
\begin{center}
    $y = 199.87-3.01\cdot x_{n1}+0.11\cdot x_{n2}+0.21\cdot x_{n3}-0.21\cdot x_{n1}\cdot x_{n2}-0.23\cdot x_{n1}\cdot x_{n3}-$
    $-0.01\cdot x_{n2}\cdot x_{n3}+0.002\cdot x_{n1}\cdot x_{n1}\cdot x_{n3}+0.39\cdot x^2_{n1}+0.01\cdot x^2_{n2}-0.001\cdot x^2_{n3}$        
\end{center}

Критерій Кохрена показав наступне значення:
\[G_p = 0.158\]

Вищезгадане значення менше за $G_T=0.3346$, а отже дисперсія однорідна.

Критерій Стьюдента показав наступні значення:
\small
\[t_0=548.2,\; t_1=1.7,\; t_2=0.5,\; t_3=0.7,\; t_4=0.3,\; t_5=0.1,\; t_6=0.4,\; t_7=0.2,\; t_8=172.1,\; t_9=171.3,\; t_{10}=171.2\]
\normalsize
Значення $t_{1,2,3,4,5,6,7}<2.306$, тому коефіцієнти рівняння регресії приймаємо
незначними при рівні значимості 0.05

Таким чином рівняння регресії має вигляд:
\[y = 199.87+0.39\cdot x^2_{n1}+0.01\cdot x^2_{n2}-0.001\cdot x^2_{n3}\]  

Також було знайдено натуралызовані коефіцєнти рівнняння регресії:
\[y = 199.87+0.39\cdot x^2_{1}+0.01\cdot x^2_{2}-0.001\cdot x^2_{3}\]  

Останнім кроком була перевірка адекватносі моделі за допомогою критерію Фішера:
\[F_p = 2.23\]

Оскільки $F_p<2.31$, отже рівняння регресії адекватно оригіналу при рівні 
значимості 0.05

\section{Висновки}
В ході даної лабораторної роботи було проведено трьохфакторний експеримент з 
використанням рівняння регресії з урахуванням квадратичних членів
(центральний ортогональний композиційний план). Було перевірено однорідність десперсії
за критерієм Кохрена. Після цього було знайдено натуралізовані коефіцієнти, та визначено
значимість коефіцієнтів за допомогою критерію Стьюдента, який показав, що один з 
коефіцієнтів є незначним. Адекватность рівняння оригіналу було перевірено за допомогою
критерію Фішера, який показав, що рівнняння є адекватним оригіналу.
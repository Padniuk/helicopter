\section{Лістинг програми}

\lstinputlisting[language=Python]{../4.py}

\newpage

\section{Результат роботи програми}

Нормована матриця планування:
\begin{center}
    \begin{tabular}{|c|c|c|c|c|c|c|c|c|c|c|}
        \hline
         & $X_{N1}$ & $X_{N2}$ & $X_{N3}$ & $X_{N1}\cdot X_{N2}$ & $X_{N1}\cdot X_{N3}$ & $X_{N2}\cdot X_{N3}$ &$X_{N1}\cdot X_{N2}\cdot X_{N3}$ & $Y_{1}$ & $Y_{2}$ & $Y_{3}$ \\ 
        \hline
        1& 1.0 & -1.0 & 1.0 & -1.0 & 1.0 & -1.0 & -1.0 & 223 & 211 & 212\\
        2& 0.58 & -0.8 & -0.29 & -0.46 & -0.17 & 0.23 & 0.13 & 244 & 217 & 242\\
        3& 0.56 & 0.8 & 0.57 & -0.45 & -0.3192 & 0.46 & -0.26 & 214 & 241 & 258\\
        4& 0.51 & 1.0 & -0.57 & 0.51 & -0.29 & -0.57 & -0.29 & 229 & 241 & 231\\
        5& -1.0 & -1.0 & 0.43 & 1.0 & -0.43 & -0.43 & 0.43 & 250 & 251 & 229\\
        6& 0.8 & 0.2 & 0.57 & 0.16 & 0.456& 0.11 & 0.09 & 223 & 241 & 253\\
        7& -0.19 & 1.0 & -1.0 & -0.19 & 0.19 & -1.0 & 0.19 & 248 & 226 & 214\\
        8& -0.33 & 0.93 & 1.0 & -0.30 & -0.33 & 0.93 & -0.30 & 247 & 245 & 213\\
        \hline
    \end{tabular}
\end{center}

Далі було знайдено нормовані коефіцієнти рівняння:
\[y = 359.14-0.04\cdot x_{n1}-2.41\cdot x_{n2}-3.23\cdot x_{n3}-0.01\cdot x_{n1}\cdot x_{n2}-0.01\cdot x_{n1}\cdot x_{n3}+0.06\cdot x_{n2}\cdot x_{n3}+0.0006\cdot x_{n1}\cdot x_{n1}\cdot x_{n3}\]

Критерій Кохрена показав наступне значення:
\[G_p = 0.266\]

Вищезгадане значення менше за $G_T=0.767$, а отже дисперсія однорідна.

Критерій Стьюдента показав наступні значення:
\[t_0=737.3,\; t_1=64.76,\; t_2=107.91,\; t_3=156.40,\; t_4=57.43,\; t_5=0.98,\; t_6=108.14,\; t_7=84.69\]

Значення $t_5<2.306$, тому коефіцієнт рівняння регресії приймаємо
незначними при рівні значимості 0.05

Таким чином рівняння регресії має вигляд:
\[y = 359.14-0.04\cdot x_{n1}-2.41\cdot x_{n2}-3.23\cdot x_{n3}-0.01\cdot x_{n1}\cdot x_{n2}+0.06\cdot x_{n2}\cdot x_{n3}+0.0006\cdot x_{n1}\cdot x_{n1}\cdot x_{n3}\]

Також було знайдено натуралызовані коефіцєнти рівнняння регресії:
\[y = 233.45+20.51\cdot x_{n1}+34.17\cdot x_{n2}+49.52\cdot x_{n3}-18.19\cdot x_{n1}\cdot x_{n2}+0.312\cdot x_{n2}\cdot x_{n3}-26.82\cdot x_{n1}\cdot x_{n1}\cdot x_{n3}\]

Останнім кроком була перевірка адекватносі моделі за допомогою критерію Фішера:
\[F_p = 0.54\]

Оскільки $F_p<4.5$, отже рівняння регресії адекватно оригіналу при рівні 
значимості 0.05

\section{Висновки}
В ході даної лабораторної роботи було проведено трьохфакторний експеримент з 
використанням рівняння регресії з урахуванням взаємодії. Було перевірено однорідність десперсії
за критерієм Кохрена. Після цього було знайдено натуралізовані коефіцієнти, та визначено
значимість коефіцієнтів за допомогою критерію Стьюдента, який показав, що один з 
коефіцієнтів є незначним. Адекватность рівняння оригіналу було перевірено за допомогою
критерію Фішера, який показав, що рівнняння є адекватним оригіналу.